\documentclass[11pt,a4paper]{article}
\usepackage[utf8]{inputenc}
\usepackage[french]{babel}
\usepackage[T1]{fontenc}
\usepackage{amsmath}
\usepackage{amsfonts}
\usepackage{amssymb}
\usepackage{graphicx}
\usepackage{xcolor}
\usepackage[left=2cm,right=2cm,top=1cm,bottom=1.4cm]{geometry}
\author{Lucas RODRIGUEZ}
\date{21 mars 2019 - 16h30}
\title{Escapade algorithmique avec Fibonacci}
\begin{document}
\maketitle

\section*{Plan général}
\begin{itemize}
\item Problématique générale
\item Sommaire (5 axes)
\item \textbf{1ère idée (Approche naïve)} : formule de récurrence (un terme de la suite résulte de la somme des 2 précédents)
\item Implémentation avec liste (Complexité temporelle linéaire, Complexité spatiale linéaire)
\item Implémentation sans liste (Complexité temporelle linéaire, Complexité spatiale constante)
\item \textbf{2ème idée} : formule explicite (suite de fibo $\Longleftrightarrow$ suite récurrente d'ordre 2, polynôme caractéristique dont les racines sont le nombre d'or et son conjugué). On obtient une formule explicite de $(F_n)_{n\in \mathbb{N})}$ \textcolor{red}{[Formule de Binet]}
\item Implémentation (Permet de calculer $F_n$ sans avoir les termes précédents MAIS nécessite une exponentiation coûteuse en mémoire)
\item \textbf{3ème idée} : Récursivité (cas de base + propagation)
\end{itemize}
Programme récursif : un processus, qui dépend de paramètres et qui fait appel à ce même processus sur d'autres arguments plus « simples ». Autrement dit, c'est un algorithme qui s'invoque lui-même
\begin{itemize}
\item Meilleure flexibilité MAIS pas performant (calcule plusieurs fois des termes)
\item Amélioration possible au niveau de la mise en cache (pour limiter les calculs inutiles). Optimisation du temps possible au détriment de l'espace et inversement
\item Comparaison des 2 méthodes (iter et rec) : linéaire \& exponentielle
$\Rightarrow$ Récursivité pas adapté au calcul des fibos (comportement chaotique)
\item \textbf{4ème idée} : On repart de la formule de récurrence et on pose le vecteur $X_n$ et la matrice carrée $A$.
Par analogie avec la formule de récurrence, on obtient la suite de matrice $(X_n)_{n\in\mathbb{N}}$ définie par récurrence de raison $A$ d'où cette formule explicite : $\forall n \in \mathbb{N}, X_n = A^n*X_0$.
Pour trouver le nombre de Fibonacci souhaité, on a : \[A^n = 
\begin{pmatrix}
   F_{n-1} & F_n \\
   F_n & F_{n+1}
\end{pmatrix}\]
Pour exponentier rapidement et de manière performante la matrice $A$, on décide d'utiliser la méthode d'expo rapide qui permet de réduire considérablement le nombre de multiplications effectuées.
\item Au lieu d'utiliser une exponentiation naïve, on décide d'implémenter une version récursive avec ces 2 éléments remarquables. La complexité de cet algorithme est logarithmique.
\item On l'applique maintenant au produit matriciel !! et on en conclut sur la complexité logarithmique de \verb+fibo_exporap+.
\item \textbf{Comparaison des capacités} : On a à ce stade 3 algorithmes qui fonctionne. L'un d'eux, la version récursive peut être éliminé avant les tests unitaires : en effet, il ne satisfait pas les contraintes de complexité imposées. (on rencontre un Overflow vers $n = 10^3$).
\item \textbf{Graphique}
\item \textcolor{red}{\textbf{Conclusion} : La version avec l'expo rapide est la plus performante (plus rapide et moins coûteuse en espace mémoire)}
\item \textbf{Application} Les nbres de Fibo interviennent dans l'analyse de l'algorithme d'Euclide. Ce dernier permet de calculer le PGCD de 2 entiers notés ici $a$ et $b$. Voici 2 versions : l'une itérative, l'autre récursive \\
Ils effectuent le même nombre d'opérations élémentaires !! Le th de Lamé nous indique que le nbre de div.euc. réalisées par l'algo est $\leq$ à 5 fois le nbre de chiffres du plus petit nbre entre $a$ et $b$
\item Le pgcd de 2 fibos consécutifs est égal à 1 et nécessite exactement n div.euc.
\item \textcolor{red}{\textbf{Conclusion} : Les nbres de Fibo sont des indicateurs pour la complexité (nbres d'opérations à effectuer) de l'algo d'Euclide.}
\end{itemize}
\end{document}